\chapter{Conclusion}\label{chap:Conclusion}

During this thesis, we have describe the implemenation and optimizations we develope in our runtime and compiler to support the addtional parallel features listed in the Fortran language Technical Specification Draft. Our focus is on the \textit{team} construct and collective procedures. The contribution of this work are summerized as following:
\begin{itemize}
\item We developed the first implementation of the anticipated team features expected to be added to Fortran 2015, in addition to implementing support for collective. We demostrated the effectiveness of these new features in the CG benchmark from the NAS Parallel Benchmark suite
\item We evaluate the language features using the microbenchmark to show the effectiveness of these new features. 
 %??
\end{itemize}

\section{Future Work}
In this thesis, we have discuss the basic implemenation of \textit{team} construct and collective operation. According to the technical specification draft, we still need to implement several more syntax, including \textit{image selector} and \textit{failed images}. We have discussed the implemenation of these features when we made design choices. 

Furthermore, in\cite{upc-teams} and later discussion about hierarchical construct in PGAS language, we have seen some cases where the team analysis can be benefit. We also consider this as part of future work that boost the power of language-based PGAS implementation. 
%\section{Summary of Contributions}
%
%\subsection{Image Segmentation}
%
%\subsection{Cardiac Morphology and Function}
%
%\subsection{Coronary Artery Shape-Motion Analysis}
%
%%%%%%%%%%%%%%%%%%%%%%%%%%%%%%%%%%%%%%%%%%%%%%%%%%%%%%%%%%%%%%%%
%\section{Progression and Scope for Future Work}
%
%
%\subsection{Algorithm for the Automatic LV Blood Pool Segmentation
%from Short-Axis Dual-Contrast MR Data}
%
%%%%%%%%%%%%%%%%%%%%%%%%%%%%%%%%%%%%%%%%%%%%%%%%%%%%%%%%%%%%%%%%
%\subsection{Algorithm for the Automatic Delineation of Myocardial
%Contours in Short-Axis Cardiac Cine-bFFE MR Sequences}
%
%
%%%%%%%%%%%%%%%%%%%%%%%%%%%%%%%%%%%%%%%%%%%%%%%%%%%%%%%%%%%%%%%%
%\subsection{Algorithm for the Automatic Computation of EF from the Short-Axis Cardiac Cine-bFFE MR Sequences}
%
%
%%%%%%%%%%%%%%%%%%%%%%%%%%%%%%%%%%%%%%%%%%%%%%%%%%%%%%%%%%%%%%%%
%\subsection{Computational Framework for the 4D
%Shape-Motion Analysis of the LAD}
%
%
%%%%%%%%%%%%%%%%%%%%%%%%%%%%%%%%%%%%%%%%%%%%%%%%%%%%%%%%%%%%%%%%
%\subsection{Future Work}
%
%%%%%%%%%%%%%%%%%%%%%%%%%%%%%%%%%%%%%%%%%%%%%%%%%%%%%%%%%%%%%%%%
